%---------------------------------------------------------------------------------------------------
% Einf�hrung
%---------------------------------------------------------------------------------------------------
\newpage
%\part{Anfang}
\chapter{Introduction}\label{cap:Ein}

% Write the Introduction
Since its market introduction in 2012, the Raspberry Pi has become one of the most distributed single-board computers for didactic usage. Due to many application possibilities, the Pi found its way into many projects in University. The development of an Audio-HAT, providing professional analog audio in- and outputs by Sebastian Albers\,\cite{Albers:2017}, extends the Pi with the capability of digital signal processing.

Digital effect units for electric guitars are manipulating the audio signal to improve the users sound experience.
These units are placed in the middle of the signal chain between the guitar output and amplifier input.
A Pi in combination with the Audio-HAT can fulfil the task of digital signal processing.
Due to the several interfaces, the audio processing can be controlled by the user without having a mouse or keyboard attached to the Pi.

The goal of this thesis is to develop a fully working multi-effects unit based on the Pi working as an embedded system. In addition to that, the unit should consist of an appropriate preamp module to adjust the guitar signal and a user interface module for the system controls.

After the theoretical part and the declaration of requirements, the design and implementation are described in a detailed way. Finally, the device is tested with regard to the requirements.




 

