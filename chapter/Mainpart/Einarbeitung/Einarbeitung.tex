%---------------------------------------------------------------------------------------------------
% Einarbeitung
%---------------------------------------------------------------------------------------------------
\newpage
%\part{Anfang}
\chapter{Einarbeitung}
F\"ur den Praktikumsbeginn habe ich mich in Absprache mit meinem Portrix Betreuer Knud M\"uller auf folgende Themen vorbereitet:
\begin{itemize}
\item Linux (Kubuntu)
\item Html5/CSS
\item Json
\item Android Studio
\end{itemize}

W\"ahrend meiner Portrix-Zeit habe sich die Eindr\"ucke vertieft und wurden um folgende Aspekte erweitert:

\begin{itemize}
\item Git
\item Java
\item JavaScript
\end{itemize}

Die generelle Informationsbeschaffung erfolgte in den meisten f\"allen \"uber das Internet. Besonders hilfreich waren dabei:

\begin{itemize}
\item stackoverflow
\item Android Developers
\item Udemy (Online Kursplattform)
\item CodingInFlow (Youtube Channel)
\item CodingWithMitch (Youtube Channel)
\end{itemize}


\section{Android Studio}
Android Studio (AS) bietet eine Entwicklungsumgebung die sowohl Java als auch Kotlin unterst\"utzt. Da die Android Abteilung bei portrix.net ausschlie\ss{}lich mit Java programmiert, bestand kein Bedarf Kotlin zu lernen.
Ein Android Projekt besteht aus vielen Komponenten, welche als Resultat eine Application (App) generieren. Eine einfache "Hello World" App besteht zum Beispiel aus der MainActivity.java welche die programmatische Logik enth\"alt und der dazugeh\"origen activity\_main.xml die das graphische Layout definiert.
Ein weitere wichtiger Bestandteil von AS ist der integrierte Emulator der nach den spezifischen Anforderungen des Projekts gestaltet werden kann und zu Testzwecken als ein simuliertes Mobiltelefon auf dem PC fungiert. In der professionellen Software Entwicklung ist es allerdings unerl\"asslich auf einem realen Testdevice (Mobiltelefon oder Tablet) zu testen, da sich Abweichungen zwischen Emulator und Mobiltelefon kaum vermeiden lassen und der haptische Aspekt nur auf dem Device getestet werden kann.
Die M\"oglichkeit AS Projekte mit Versionskontrollsystemen (VCS) zu nutzen spielt ebenfalls eine gro\ss{}e Rolle, da mehrere Entwickler gleichzeitig an den Projekten arbeiten, diese gut aufteilen und dokumentieren k\"onnen. Bei portrix.net wird mit GitLab gearbeitet, worauf im weiteren Verlauf eingegangen wird.

\section{Java}
Java als Objekt orientierte Programmiersprache (OOP) ist das Fundament f\"ur das arbeiten mit Android Studio bei portrix.net. Die Grundlagen der Objekt orientierten Programmierung die an der HAW vermittelt wurden, konnten dementsprechend erweitert und vertieft werden. Dies geschah zu beginn haupts\"achlich durch Kurse der Online-Plattform Udemy zum Thema OOP, wie zum Beispiel "Java leicht gemacht - Der umfassende Java Einsteigerkurs A-Z", oder 
%https://www.udemy.com/programmieren-lernen-mit-java-ein-kurs-fur-einsteiger/
auch "Der Komplette Android 8 Entwickler Kurs - Erstelle 20+ Apps" 
%https://www.udemy.com/der-komplette-android-8-entwickler-kurs/ 
Mit dem dadurch erlangtem Basiswissen, war es m\"oglich bei der Weiterentwicklung der Firmen Apps beteiligt zu sein.

\section{Linux (Kubuntu)}
Die Firma portrix.net benutzt auf ihren Firmenrechnern Kubuntu als Betriebssystem. Es war also notwendig ein Grundwissen an Terminal-Befehlen (Bash) aufzubauen. Dies wurde als Vorbereitung vor dem Praktikum durch das Youtube-Tutorial "The Complete Linux Course: Beginner to Power User!"\footnotemark geleistet und war besonders in Verbindung mit der Nutzung von GitLab sehr vorteilhaft. \footnotetext{URL: https://www.youtube.com/watch?v=wBp0Rb-ZJak} 

\section{Git}
Wie zuvor erw\"ahnt wird GitLab als VCS-Tool verwendet. "GitLab ist eine Webanwendung zur Versionsverwaltung f\"ur Softwareprojekte auf Basis von Git"\footnotemark \footnotetext{URL: https://de.wikipedia.org/wiki/GitLab} 
Um Git zu nutzen speichert man sein Inkrement auf dem lokalen PC mit dem commit-Befehl und der commit-Message welche die \"Anderung beschreibt. Anschlie\ss{}end werden alle gew\"unschten commits mit dem push-Befehl ins Online-Repository gespeichert. Dadurch k\"onnen mehrere Personen gleichzeitig an einem Projekt arbeiten. Es bietet auch die M\"oglichkeit neue features in einer parallel laufenden Verzweigung (Branch) zu entwickeln und so die Funktionalit\"at der bisherigen Version zu erhalten. 



\section{Json}
\section{JavaScript}
\section{Html5/CSS}

udemy

