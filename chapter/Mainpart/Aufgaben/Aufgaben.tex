%---------------------------------------------------------------------------------------------------
% Aufgaben
%---------------------------------------------------------------------------------------------------
\newpage
%\part{Anfang}
\chapter{Aufgaben}
Aufgrund der in der Einarbeitung erworbenen Kenntnisse war es m\"oglich an folgenden Projekten zu partizipieren. Im Fokus stand die H2-Mobility App, mit einer geringeren Priorit\"at folgte die LFI App und zum Schluss gab es noch ein Projekt in dem eine sogenannte Launcher App entwickelt werden sollte. 

\section{Theorie}
Die vielf\"altigen  M\"oglichkeiten und Eigenschaften die Android Studio bietet k\"onnen im Rahmen dieses Praktikumsberichts nicht im Detail beschrieben werden. Daher wird hier nur auf die f\"ur die Aufgabenbew\"altigung wesentlichen Aspekte eingegangen. Dazu z\"ahlen besonders Activities und Fragments. Im folgenden wird dazu ein kurzer \"Uberblick pr\"asentiert und auschlie\ss{}end auf die verschiedenen Apps eingegangen.

\subsection{Activity}
Eine Activity dient als Fenster in dem User Interface Elemente (UI) platziert sind um mit dem Benutzer zu interagieren. Sie f�llt beinahe den gesamten Bildschirm des Ger�tes aus und implementiert verschiedene Methoden die dem Lifecycle der Activity dienen und um die Activity mit der zugeh�rigen .xml Datei zu verbinden. 

\subsection{Fragment}
Fragments sind ein Teil der UI von der App und k�nnen in einer Activity platziert werden. Sie geh\"oren demnach zu einer Activity. Es ist auch m�glich mehrere Fragmente in einer Activity zu platzieren. Der aktuell beste Ansatz ist es eine Activity zu haben und den Rest �ber Fragmente zu realisieren. Abbildung (\ref{fig:lifecycleOverview}) zeigt eine \"Ubersicht zum Lifecycle von Fragment und Activity.
%https://stackoverflow.com/questions/20306091/dilemma-when-to-use-fragments-vs-activities

\begin{figure}[H]
	\centering \includegraphics[width=0.8\textwidth]{lifecycleOverview.png}
	\caption[lifecycleOverview]{Prototyp der InVision Vorgabe. Feedback Button f\"ur konkret ausgew\"ahlte Tankstelle}
	\label{fig:lifecycleOverview}
\end{figure}
%https://developer.android.com/reference/android/app/Activity#ActivityLifecycle
%https://docs.microsoft.com/de-de/xamarin/android/platform/fragments/creating-a-fragment

% ----------------------------
\section{H2 Mobility App}
Die App ist f\"ur Besitzer von Wasserstoffautos, also Autos mit einer Brennstoffzelle, gedacht. "Die Brennstoffzelle wird sich durchsetzen"\footnotemark \footnotetext{URL: http://www.spiegel.de/auto/aktuell/wasserstoffauto-die-brennstoffzelle-wird-sich-durchsetzen-a-1235431.html}, ist sich Toyota-Motorenentwickler Gerald Killmann sicher. Daher hat die H2 Mobility Deutschland GmbH \& Co.KG die Firma portrix.net mit der Entwicklung einer App beauftragt die den Nutzern unter anderem das europ\"aische Tankstellennetz anzeigt, die Route zur n\"achstgelegenen Tankstelle anzeigt, eine Feedback f\"ur die Tankstellenbetreiber, Schulungsvideos zum Tankvorgang und viele weitere Funktionen bietet.
Die Entwicklung der App erfolgt nach R\"ucksprache mit dem Vertreter von H2 Mobility und dem f\"ur das Layout verantwortlichen Designer und wird sowohl f\"ur iOS als auch f\"ur Android bei portrix.net programmiert. Features und design-Prototypen werden \"uber das digitale Produkt Design Tool InVision an die Programmierer vermittelt und anschlie\ss{}end umgesetzt. Aufgrund der hohen Komplexit\"at und des Umfangs den die App mittlerweile angenommen hat sich das Erscheinungsbild zwischenzeitlich stark ver\"andert. Das organische anwachsen der Funktionalit\"at ist am Versionskontrollbaum gut zu erkennen, da das Projekt beinahe 20000 commits enth\"alt.

\subsection{Feedback}
\"Uber InVision kam die Vorgabe einen Prototyp zu entwickeln, der es erm\"oglicht einer konkret ausgew\"ahlten Tankstelle per Knopfdruck ein Feedback zu geben.

\begin{figure}[H]
	\centering \includegraphics[width=0.8\textwidth]{feedbackPrototyp.png}
	\caption[FeedbackPrototyp]{Prototyp der InVision Vorgabe. Feedback Button f\"ur konkret ausgew\"ahlte Tankstelle}
	\label{fig:FeedbackPrototyp}
\end{figure}
%\footnotetext{URL: https://www.guitarchalk.com/wp-content/uploads/2017/07/electric-guitar-parts.jpg [cited 22 August 2018]}


In der bisherigen Version war es m\"oglich \"uber eine Knopf in der FuelStationsMap.java (MainActivity) das HelpAndFeedback.java (Fragment) aufzurufen und von dort in ein allgemeine Feedback.java (Fragment) zu kommen. Dort l\"asst sich aus einem Spinner, der jede Tankstelle auflistet, eine Tankstelle ausw\"ahlen. Es war also naheliegend den Code zu modifizieren, um aus der FuelStationDetail.java (Activity) direkt in das Feedback.java zu wechseln und die angew\"ahlte Tankstelle direkt im Spinnerfeld anzuzeigen.
Hierzu mussten diverse \"Anderungen vorgenommen werden, da das Feedback.java der darunterliegenden FuelStationsMap.java geh\"ort und in der OnCreate() Methode des Feedback.java eine FuelStationsMap Variable als Context initialisiert wird (\ref{code:onCreate}). Dieser Context gew\"ahrleistet unter anderem das der Spinner im Feedback Fragment mit Tankstellen Daten gef\"ullt werden kann. Der Context macht es allerdings unm\"oglich das Feedback.java von einer weiteren Activity, also der FuelStationDetail.java gestartet wird. Es war daher notwendig die FuelStationDetail.java (Activity) als Fragment nachzubauen. 
Der gew\"unschte workflow wird in Abb.(\ref{fig:inVisionFeedbackPrototyp}) dargestellt.

\begin{figure}[H]
	\centering \includegraphics[width=0.8\textwidth]{inVisionFeedbackPrototyp.png}
	\caption[inVisionFeedbackPrototyp]{Workflow der neuen Feedback Funktion}
	\label{fig:inVisionFeedbackPrototyp}
\end{figure}


\fbox{
\lstinputlisting[label={code:onCreate} ,caption={onCreate() Methode der Feedback.java},captionpos=b, language = java,  numbers = left]{program/Feedback.java}
}

Das Layout einer App wird in der zur Activity bzw. Fragment zugeh\"origen .xml Datei definiert. In diesem Fall konnte der meiste Inhalt aus der activity\_fuel\_station\_detail.xml in die neue fragment\_fuel\_station\_detail.xml kopiert und um das gew\"unschte neue Feature erg\"anzt werden (\ref{code:feedbackXml}). 


\fbox{
\lstinputlisting[label={code:feedbackXml} ,caption={Designelement Feedback Button},captionpos=b, language = xml,  numbers = left]{program/fragment_fuel_station_detail.xml}
}

Der Transfer (1 -> 2a) erfolgt durch anklicken eines Tankstellen Markers. Die bisherige Logik rief die FuelStationDetail.java (Activity) auf und musste durch eine Fragment Transaktion ersetzt werden. Zu sehen in Abb.(\ref{code:markerTransaction})
Beim Aufruf von onMarkerClick() wird der angeklickte Marker als Argument �bergeben. Es wird �ber die MapHelper Klasse eine FuelStation Variable station zugewiesen und dessen idx, wenn die station ungleich null ist in ein Bundle gespeichert. Dieses Bundle b wird einem neuen FuelStationDetail Fragment als Argument �bergeben und anschliessend wird mit dem SupportFragmentManager die Fragment Transaktion zum FuelStationDetail mit ausgew�hlter FuelStation.

\fbox{
\lstinputlisting[label={code:markerTransaction} ,caption={Wechsel von FuelStationsMap zu FuelStationDetail},captionpos=b, language = java,  numbers = left]{program/MarkerTransaction.java}
}


Um �ber die Detailansicht der Tankstelle zum gew�nschten Feedback (2a -> 3) zu gelangen muss die ID des FrameLayout lyFeedbackPin deklariert und initialisiert werden und anschlie\ss{}end \"uber einen onClickListener() angeklickt werden. Siehe Abb.(\ref{code:transaction}). Das Prozedere ist dem vorangegangenem \"ahnlich. Der unterschied besteht darin das in der onClick() Methode das Bundle b �ber getArguments() den Marker �bergeben bekommt und an das FeedbackFragment weiterreicht, damit dort das Spinner field menu den Tankstellen namen vorausw�hlen kann. Die eigentliche Transaktion von FuelStationDetail zu Feedback funktioniert nach dem gleichen Muster wie zuvor beschrieben.

\fbox{
\lstinputlisting[label={code:transaction} ,caption={Wechsel von FuelStationDetail zu Feedback},captionpos=bl, language = java,  numbers = left]{program/Transaction.java}
}

%--------------
%design vorgabe von Invision
%direktes Feedback im Fragment
%-----------
\subsection{Opt-Out}
Um das Nutzerverhalten der App zu verbessern werden s\"ammtliche interaktionen des Benutzers mithilfe von Google Analytics registriert. Aufgrund des Europ\"aischen Datenschutzgesetzes muss es dem Benutzer m�glich sein dies zu deaktivieren. Dazu bietet Google Analytics die sogenannte Opt-Out Funktion an. Der "Google Analytics deaktivieren"-Button befindet sich im Impressum (Abb.(\ref{fig:impressum})). Das Impressum ist ein WebView indem die Url des Impressums der H2.Live Webseite aufgerufen und angezeigt wird. Die HTML (\ref{fig:impressum}) enth�lt JavaScript Elemente die in der App ausgelesen und interpretiert werden, wenn der Button gedr�ckt wird.

\begin{figure}[H]
	\centering \includegraphics[width=0.4\textwidth]{impressum.png}
	\caption[impressum]{Impressum mit Google Analytics deaktivieren Button}
	\label{fig:impressum}
\end{figure}

\fbox{
\lstinputlisting[label={code:optout} ,caption={Html des Impressums},captionpos=b, language = html,  numbers = left]{program/optout.html}
}

\fbox{
\lstinputlisting[label={code:webview},caption={Interpretation von JavaScript in Android Studio},captionpos=b, language = html,  numbers = left]{program/webview.java}
}


% ----------------------------
\section{LFI App}
super fotografie app
\subsection{Magazin Overview}
neugestaltung der Magazin ansicht
\subsection{Magazin Preview}
Magazin vorschau mit 10 beispielseiten

% ----------------------------
\section{Launcher App}
angesagte Spiegel Tablet