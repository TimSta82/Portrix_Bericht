%---------------------------------------------------------------------------------------------------
% Einarbeitung
%---------------------------------------------------------------------------------------------------
\newpage
%\part{Anfang}
\chapter{Einarbeitung}
F\"ur den Praktikumsbeginn habe ich mich in Absprache mit meinem Portrix Betreuer Knud M\"uller auf folgende Themen vorbereitet:
\begin{itemize}
\item Linux
\item Html5/CSS
\item Json
\item Android Studios
\end{itemize}

W\"ahrend meiner Portrix-Zeit habe sich die Eindr\"ucke vertieft und wurden um folgende Aspekte erweitert:

\begin{itemize}
\item Git
\item Java
\item JavaScript
\end{itemize}

Die generelle Informationsbeschaffung erfolgte in den meisten f\"allen \"uber das Internet. Besonders hilfreich waren dabei:

\begin{itemize}
\item stackoverflow
\item Android Developers
\item Udemy (Online Kursplattform)
\item CodingInFlow (Youtube Channel)
\item CodingWithMitch (Youtube Channel)
\end{itemize}


\section{Android Studios}
Android Studios (AS) bietet eine Entwicklungsumgebung die sowohl Java als auch Kotlin unterst\"utzt. Da die Android Abteilung bei portrix.net ausschlie\ss{}lich mit Java programmiert, bestand kein Bedarf Kotlin zu lernen.
Ein Android Projekt besteht aus vielen Komponenten, welche als Resultat eine Application (App) generieren. Eine einfache "Hello World" App besteht zum Beispiel aus der MainActivity.java welche die programmatische Logik enth\"alt und der dazugeh\"origen activity\_main.xml die das graphische Layout definiert.
Ein weitere wichtiger Bestandteil von AS ist der integrierte Emulator der nach den spezifischen Anforderungen des Projekts gestaltet werden kann und zu Testzwecken als ein simuliertes Mobiltelefon auf dem PC fungiert. In der professionellen Software Entwicklung ist es allerdings unerl\"asslich auf einem realen Testdevice (Mobiltelefon oder Tablet) zu testen, da sich Abweichungen zwischen Emulator und Mobiltelefon kaum vermeiden lassen und der haptische Aspekt nur auf dem Device getestet werden kann.
Die M\"oglichkeit AS Projekte mit Versionskontrollsystemen (VCS) zu nutzen spielt ebenfalls eine gro\ss{}e Rolle, da mehrere Entwickler gleichzeitig an den Projekten arbeiten, diese gut aufteilen und dokumentieren k\"onnen. Bei portrix.net wird mit GitLab gearbeitet, worauf im weiteren Verlauf eingegangen wird.

\section{Java}

\section{Git}
\section{Linux}
\section{Json}
\section{JavaScript}
\section{Html5/CSS}

udemy

