%---------------------------------------------------------------------------------------------------
% Aufgaben
%---------------------------------------------------------------------------------------------------
\newpage
%\part{Anfang}
\chapter{Aufgaben}
Aufgrund der in der Einarbeitung erworbenen Kenntnisse war es m\"oglich an folgenden Projekten zu partizipieren. Im Fokus stand die \textbf{H2.LIVE} App, welche im weiteren Verlauf exemplarisch und detailiert beschrieben wird. Mit einer geringeren Priorit\"at folgte die \textbf{LFI} App und zum Schluss gab es noch ein Projekt in dem eine sogenannte \textbf{Launcher} App entwickelt werden sollte. Diese beiden Apps werden nur kurz beschrieben. 

\section{Theorie}
Die vielf\"altigen  M\"oglichkeiten und Eigenschaften die Android Studio bietet k\"onnen im Rahmen dieses Praktikumsberichts nicht im Detail beschrieben werden. Daher wird hier nur auf die f\"ur die Aufgabenbew\"altigung wesentlichen Aspekte eingegangen. Dazu z\"ahlen besonders \textit{Activities} und \textit{Fragments}. Im folgenden wird dazu ein kurzer \"Uberblick pr\"asentiert und auschlie\ss{}end auf die verschiedenen Apps eingegangen.

\subsection{Activity}
Eine \textit{Activity} dient als Fenster in dem \textit{User Interface} Elemente (UI) platziert sind, um mit dem Benutzer zu interagieren. Sie f�llt den gesamten Bildschirm des Ger�tes aus und implementiert verschiedene Methoden die dem \textit{Lifecycle} der \textit{Activity} dienen und um die \textit{Activity} mit der zugeh�rigen .xml Datei zu verbinden.\footnote{vgl: https://developer.android.com/reference/android/app/Activity}

\subsection{Fragment}
\textit{Fragments} sind ein Teil der UI der App und k�nnen in einer \textit{Activity} platziert werden. Sie geh\"oren demnach zu einer \textit{Activity}. Es ist auch m�glich mehrere \textit{Fragments} in einer \textit{Activity} zu platzieren. Der aktuell beste Ansatz ist es eine \textit{Activity} zu haben und den Rest �ber \textit{Fragments} zu realisieren\footnote{vgl: https://stackoverflow.com/questions/20306091/dilemma-when-to-use-fragments-vs-activities}. Abbildung (\ref{fig:lifecycleOverview}) zeigt eine \"Ubersicht zum \textit{Lifecycle} von \textit{Fragment} und \textit{Activity}.

\begin{figure}[H]
	\centering \includegraphics[width=0.8\textwidth]{lifecycleOverview.png}
	\caption[lifecycleOverview]{Lifecycle von Activity und Fragment\footnote{https://docs.microsoft.com/de-de/xamarin/android/platform/fragments/creating-a-fragment}}
	\label{fig:lifecycleOverview}
\end{figure}


% ----------------------------
\section{H2.LIVE}
Die App ist f\"ur Besitzer von Wasserstoffautos, also Autos mit einer Brennstoffzelle, gedacht. "Die Brennstoffzelle wird sich durchsetzen"\footnote{http://www.spiegel.de/auto/aktuell/wasserstoffauto-die-brennstoffzelle-wird-sich-durchsetzen-a-1235431.html}, ist sich Toyota-Motorenentwickler Gerald Killmann sicher. Daher hat die H2 Mobility Deutschland GmbH \& Co.KG die Firma portrix.net mit der Entwicklung der H2.LIVE App beauftragt, die den Nutzern unter anderem das europ\"aische Tankstellennetz und die Route zur n\"achstgelegenen Tankstelle anzeigt, ein Feedback f\"ur die Tankstellenbetreiber, Schulungsvideos zum Tankvorgang und viele weitere Funktionen bietet. Zuk\"unftig soll es auch m\"oglich sein mittels Smartphone via App am Tankterminal zu bezahlen.
Die Entwicklung der App erfolgt nach R\"ucksprache mit dem Vertreter von H2 Mobility und dem f\"ur das Layout verantwortlichen Designer und wird sowohl f\"ur iOS als auch f\"ur Android bei portrix.net programmiert. Features und design-Prototypen werden \"uber das digitale Produkt Design Tool InVision an die Programmierer vermittelt und anschlie\ss{}end umgesetzt. Aufgrund der hohen Komplexit\"at und des Umfangs den die App mittlerweile angenommen hat, hat sich das Erscheinungsbild zwischenzeitlich stark ver\"andert. Das organische anwachsen der Funktionalit\"at ist am Versionskontrollbaum gut zu erkennen, da das Projekt beinahe 20000 Commits enth\"alt.

\subsection{Feedback}
\"Uber InVision kam die Vorgabe einen Prototyp zu entwickeln, der es erm\"oglicht einer konkret ausgew\"ahlten Tankstelle per \textit{Button} ein Feedback zu geben.

\begin{figure}[H]
	\centering \includegraphics[width=0.8\textwidth]{feedbackPrototyp.png}
	\caption[FeedbackPrototyp]{Prototyp der InVision Vorgabe. Feedback Button f\"ur konkret ausgew\"ahlte Tankstelle}
	\label{fig:FeedbackPrototyp}
\end{figure}


In der bisherigen Version war es m\"oglich \"uber einen \textit{Button} in der FuelStationsMap.java (MainActivity) das HelpAndFeedback.java (\textit{Fragment}) aufzurufen und von dort in ein allgemeines Feedback.java (\textit{Fragment}) zu kommen (Siehe Abb.\ref{fig:inVisionFeedbackPrototyp}) (1 -> 2b -> 3). Dort l\"asst sich aus einem \textit{Spinner}, der jede Tankstelle auflistet, eine Tankstelle ausw\"ahlen. Es war also naheliegend den Code zu modifizieren, um aus der FuelStationDetail.java (\textit{Activity}) direkt in das Feedback.java zu wechseln und die angew\"ahlte Tankstelle im \textit{Spinner} anzuzeigen.
Hierzu mussten diverse \"Anderungen vorgenommen werden, da das Feedback.java der darunterliegenden FuelStationsMap.java geh\"ort und in der \textit{OnCreate()} Methode des Feedback.java eine FuelStationsMap Variable als \textit{Context} initialisiert wird (\ref{code:onCreate}). Dieser \textit{Context} gew\"ahrleistet unter anderem das der \textit{Spinner} im Feedback.java mit Daten der Tankstellen gef\"ullt werden kann. Der \textit{Context} macht es allerdings unm\"oglich das Feedback.java von einer weiteren \textit{Activity}, also der FuelStationDetail.java (\textit{Activity}) gestartet wird. Es war daher notwendig die FuelStationDetail.java (\textit{Activity}) als \textit{Fragment} nachzubauen. 
Der gew\"unschte Ablauf wird in Abb.\ref{fig:inVisionFeedbackPrototyp} dargestellt. Zusammenfassend l\"asst sich sagen, das ein gro\ss{} des Codes der in der \textit{onCreate()} Methode der FuelStationDetail.java (\textit{Activity}) ausgef\"uhrt wird, in die \textit{onCreateView()} Methode der FuelStationDetail.java (\textit{Fragment}) kopiert werden muss. Allerdings ist die Initialisierung der UI Elemente in der korrespondierenden .xml Datei anders zu behandeln. Das Fragment bekommt ein \textit{View} \"ubergeben und dieser \textit{View} muss den UI Elementen vorangestellt werden.

\begin{figure}[H]
	\centering \includegraphics[width=0.8\textwidth]{inVisionFeedbackPrototyp.png}
	\caption[inVisionFeedbackPrototyp]{Workflow der neuen Feedback Funktion}
	\label{fig:inVisionFeedbackPrototyp}
\end{figure}


\fbox{
\lstinputlisting[label={code:onCreate} ,caption={onCreate() Methode der Feedback.java},captionpos=b, language = java,  numbers = left]{program/Feedback.java}
}

Das Layout einer App wird in der zur \textit{Activity} bzw. \textit{Fragment} zugeh\"origen .xml Datei definiert. In diesem Fall konnte der meiste Inhalt aus der \textit{activity\_fuel\_station\_detail.xml} in die neue \textit{fragment\_fuel\_station\_detail.xml} kopiert und um das gew\"unschte neue Feature erg\"anzt werden (Abb.\ref{code:feedbackXml}). 


\fbox{
\lstinputlisting[label={code:feedbackXml} ,caption={Designelement Feedback Button},captionpos=b, language = xml,  numbers = left]{program/fragment_fuel_station_detail.xml}
}

Der Transfer (1 -> 2a) erfolgt durch anklicken eines Tankstellen \textit{Markers}. Die bisherige Logik rief die FuelStationDetail.java (\textit{Activity}) auf und musste durch eine \textit{FragmentTransaction} ersetzt werden. Zu sehen in Abb.\ref{code:markerTransaction}.
Beim Aufruf von \textit{onMarkerClick()} wird der angeklickte \textit{Marker} als Argument �bergeben. Es wird �ber die MapHelper Klasse eine FuelStation Variable \textbf{station} zugewiesen und dessen Property \textbf{idx}, wenn die \textbf{station} ungleich null ist, in ein \textit{Bundle} gespeichert. Dieses \textit{Bundle} \textbf{b} wird einem neuen FuelStationDetail (\textit{Fragment}) als Argument �bergeben und anschlie\ss{}end wird mit dem \textit{SupportFragmentManager} die \textit{FragmentTransaction} zum FuelStationDetail mit ausgew�hlter FuelStation durchgef\"uhrt.

\fbox{
\lstinputlisting[label={code:markerTransaction} ,caption={Wechsel von FuelStationsMap zu FuelStationDetail},captionpos=b, language = java,  numbers = left]{program/MarkerTransaction.java}
}


Um �ber die Detailansicht der Tankstelle zum gew�nschten Feedback (2a -> 3) zu gelangen muss die ID des \textit{FrameLayout} \textbf{lyFeedbackPin} (Abb.\ref{code:feedbackXml} Zeile 3) deklariert und initialisiert werden und anschlie\ss{}end \"uber einen \textit{onClickListener()} angeklickt werden (Abb.\ref{code:transaction}). Das Prozedere ist dem vorangegangenem \"ahnlich. Der unterschied besteht darin das in der \textit{onClick(}) Methode das Bundle \textbf{b} �ber \textit{getArguments()} den \textit{Marker} �bergeben bekommt und an das Feedback.java weiterreicht, damit dort das \textit{Spinner field menu} den Tankstellen Namen vorausw�hlen kann. Die eigentliche Transaktion von FuelStationDetail zu Feedback funktioniert nach dem gleichen Muster wie zuvor beschrieben.

\fbox{
\lstinputlisting[label={code:transaction} ,caption={Wechsel von FuelStationDetail zu Feedback},captionpos=bl, language = java,  numbers = left]{program/Transaction.java}
}


\subsection{Opt-Out}
Um das Nutzerverhalten der App zu verbessern werden s\"amtliche Interaktionen des Benutzers mithilfe von \textit{Google Analytics} registriert. Aufgrund der Datenschutzgrundverordnung\footnote{https://www.121watt.de/online-marketing/google-analytics-datenschutz-konform/} (DSVGO) muss es dem Benutzer m�glich sein dies zu deaktivieren. Dazu bietet Google Analytics die sogenannte Opt-Out Funktion an. Der \textit{Google Analytics deaktivieren-Button} befindet sich im Impressum (Abb.\ref{fig:impressum}). Das Impressum ist ein \textit{WebView} indem die Url des Impressums der H2.Live Website aufgerufen und angezeigt wird. Die \textit{HTML} (\ref{fig:impressum}) enth�lt JavaScript Elemente die in der App ausgelesen und interpretiert werden, wenn der \textit{Button} gedr�ckt wird.  

\begin{figure}[H]
	\centering \includegraphics[width=0.4\textwidth]{impressum.png}
	\caption[impressum]{Impressum mit Google Analytics deaktivieren Button}
	\label{fig:impressum}
\end{figure}

Abbildung (\ref{code:optout}) (Zeile 7) zeigt das \textit{HTML} CodeSegment in dem der \textit{Google Analytics Deaktivieren} Button definiert wird. Ab Zeile 13 wird mit JavaScript beschrieben, was f�r die verschiedenen Betriebssysteme bei Knopfdruck passieren soll. Zeile 20 beschreibt das verhalten f�r Android und Zeile 22 f�r iOS. Es ist von entscheidender Bedeutung, dass die Klassen, Methoden und Objekte im Java\-Code die gleichen Namen haben wie im JavaScript. Damit der Java\-Code JavaScript interpretieren kann muss ein JavaScript\-Interface implementiert werden. In Abbildung (\ref{code:webview}) ist zu sehen, wie in der inneren Java Klasse GoogleAnalytics.java die Methode postMessage() aufgerufen und als Argument ein JsonObject mit dem Property \textbf{action} �bergeben wird. Wenn \textbf{action} den Wert \textbf{optOut} beinhaltet wird der \textit{Google Analytics Tracker} deaktiviert indem der Methode \textit{setTrackerOptOut(true)} als Argument �bergeben wird. Die Implementierung des JavaScript\-Interface erfolgt durch die \textit{@~Annotation} (Zeile 1) und schlie\ss{}lich durch das verbinden der Java Klasse \textit{GoogleAnalytics()} mit der JavaScript Klasse \textit{googleAnalytics} (Zeile 22).

\fbox{
\lstinputlisting[label={code:optout} ,caption={HTML des Impressums},captionpos=b, language = html,  numbers = left]{program/optout.html}
}

\fbox{
\lstinputlisting[label={code:webview},caption={Interpretation von JavaScript in Android Studio},captionpos=b, language = html,  numbers = left]{program/webview.java}
}


% ----------------------------
\newpage
\section{LFI App}
Diese App richtet sich an Fotografen mit einer Leica Kamera. Es gibt eine Version f�r Tablets und eine f�r Smartphones. Die App hat die folgenden Rubriken: 
\begin{itemize}
\item News
\item Blog
\item Galerie
\item Magazin
\item Video
\item Shop
\end{itemize}

In der Rubrik Magazine werden zum die Magazine als PDF zum kauf angeboten. �ber InVision kam der Auftrag die Ansicht eines Magazins neu zu gestalten und die Vorschau auf ein nicht gekauftes Magazin zu erm�glichen. Die Vorschau sollte Deckblatt und Inhaltsverzeichnis plus zehn Seiten des Magazins beinhalten. Aus Platzgr�nden wird die Realisierung in diesem Bericht nicht weiter beschrieben.


% ----------------------------
\section{Launcher App}
Unter einem Launcher wird beim OS Android die App verstanden die beim Starten des Smartphones aktiviert wird, zum Beispiel "Pixel Launcher". Aus dem Launcher lassen sich alle weiteren Apps starten. Android erm\"�glicht es die Launcher App zu \"andern bzw. eine eigene Launcher App zu verwenden. Diese eigene App sollte bei portrix.net entwickelt werden und nur einen gewissen "abgespeckten" Umfang an Apps beinhalten. Um eine App als Launcher zu deklarieren muss dies im \textit{Manifest} der App festgelegt werden. Zum Ende des Praxissemesters war es m�glich auf dem Startbildschirm zuvor ausgew�hlte Apps und maximal drei \textit{Widgets} anzuzeigen und zu starten. Aus Zeitgr�nden hat die Launcher App den Status einer Finger�bung nicht �berschritten und aus Platzgr�nden wird sie hier nicht weiter beschrieben.
