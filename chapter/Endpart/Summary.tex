%---------------------------------------------------------------------------------------------------
% Summary
%---------------------------------------------------------------------------------------------------
\newpage
%\part{Schluss}
%\chapter{Schlussteil}
%Ber�hmte letzte Worte...

\chapter{Summary}
% Redo Intruduction
In this thesis, the goal of a fully working multi-effect unit is accomplished.
A Raspberry Pi used in combination with the Audio-HAT performs the task of digital signal processing.\\
Three exemplary guitar effects are implemented:
A clean effect for an unchanged guitar sound, a delay effect with three adjustable parameters for a variety of echoing sounds and a distortion effect based on hard clipping.
All guitar effects are verified on the basis of measurable and hearable results.

Due to the development of a preamp module, the guitar signal is adjusted in an appropriate way.
Hence, the input signal is amplified for an optimal interconnection with the Audio-HAT providing a THD+N of 0.006\,\%.
The other way around the signal is attenuated for further processing by a guitar amplifier.

The user interface allows the guitar player the total system control.
Consisting of a bright LCD-Text display, rotary encoders and buttons the interface is suitable for stage usage.



\subsubsection{Outlook}
% Outlook: Backpanel
%			- Line out  // mention in Test
The concept of the effect unit offers a vast amount of possible extensions.\\
On the hardware side, the unit can be equipped with a power amplifier, due to the generously designed power switching supply. The back panel leaves enough space for speaker connectors. For the interconnection with other
audio devices like HIFI-Amplifiers or PC-Soundcards, a Line-Out via RCA connector sockets is reasonable.
Suitable nominal voltages on consumer audio line level and impedances are available.
Furthermore, the SW contact of the rotary encoders (high signal when the encoder is pressed) could be used for  extensions of the user controls.

The implemented software can be extended by an arbitrary number of guitar effects.
For further developments the source-code is commented at the relevant lines, highlighting good entry points. To provide a greater user-friendliness, it is reasonable to extend the user menu by displayed parameter names.
As a final step, the combination of different effects might be a good idea to achieve an effect chain.


