%---------------------------------------------------------------------------------------------------
% Analyse
%---------------------------------------------------------------------------------------------------
\newpage
\chapter{Requirements}
% fully working Prototype as standard alone device
% hearable usufull effects, measured !!
% in two sections SW HW

The goal of this thesis is to develop a multi-effects unit for electric guitars.
This device should be able to modify an electric guitar's signal according to the required 
effect. As a deliverable, the result should be hearable and measurable.
Designed as a prototype development this project should also show potential improvements for
further development. 
The requirements are split up into the hardware and software demands.

\section{Hardware}

All components of the unit should be placed in a 19-inch case designed for standardized mounting
in a 19-inch rack.
The unit is supposed to use only one power supply from the 230\,V electricity  grid.
The digital signal processing shall be done by a Raspberry Pi in combination with the Audio-HAT developed by Sebastian Albers \cite{Albers:2017}. 
For the signal in- and output, the device should be equipped with two common standard 6,3\,mm audio jacks on the front panel. That allows an easy interconnection with an electric guitar and a guitar amplifier.
In regard to limit the application range, for this development cycle only input signals from passive six-string electric guitars are to be used. Other instruments like electric basses and keyboards are excluded from testing.
In addition to that, a development of a preamp module is necessary for the adjustment from instrument level to the line level of the Audio-HAT's input. The effect unit itself shall not change the loudness of the signal from input to output.
For the user control, a suitable user interface should be mounted on the front panel.


\section{Software}

To demonstrate the audio signal processing capabilities of the Pi in combination with the Audio-HAT, three exemplary guitar effects should be implemented:
A clean, delay and distortion effect according to the hearable and measurable specifications\,(\ref{tab:guitarEffects}).
To control these effects a menu depending on the signals from the user interface should be implemented.
Required features are:

\begin{center} 
   \begin{varwidth}{4in} 
      \begin{itemize} 
      	 \item Visualize current settings on a display 
         \item Switching to another effect
         \item Change three parameters of the effect
      \end{itemize} 
   \end{varwidth} 
\end{center} 

The unit is not supposed to provide a function to combine the effects with each other.
In order to achieve a good stage performance, the maximum latency of the unit from input to output should not be more than $\Delta$t = 10\,ms.
