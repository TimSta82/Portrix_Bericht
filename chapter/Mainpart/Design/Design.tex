%---------------------------------------------------------------------------------------------------
% Design
%---------------------------------------------------------------------------------------------------
\newpage
\chapter{Design}

In this chapter, the design process of the effect unit is described.
Before the single parts of the device are developed, a thorough design phase is reasonable.
The usage of a Raspberry Pi 3 Model B in combination with the Audio-HAT is foreseen (for further explanation see section \ref{cap:TheoryPi}).
In the following sections, the choice of hardware and software in order to fulfil the requirements is explained and reasoned.


% Bild von der ersten Idee
% Placed on Amp usw..
% UML of component functions??

\section{Hardware}

\subsection{Case}
All components of the unit are supposed to be placed in one 19-inch case.
19-inch racks offer the advantage of mounting electronic devices into a standardized frame.
In the field of recording studios or stage engineering, they are commonly used.
Cases are available in different rack units $U$ defining the height.
One $U$ is specified by 1,75inch.
The used 19-inch case has $2U$ to provide enough space for the planned display (see \ref{DesignDisplay}).

% Mounted in Rack 
% 2HE for enough space for display
% Plastic for easy modification
%/ explain this choice


\subsection{Preamp Module} \label{DesignPreamp}

The input- and outputlevel of the Audio-HAT is designed for a reference voltage of 1\,V$_{\mathrm{RMS}}$ = 0\,dBV \cite[p.\,31]{Albers:2017}. As a consequence, the Audio-HAT is developed for the interaction with devices on consumer audio line level\,(see section \ref{cap:BasicsOfAudio}).
Therefore a pre-amplifier\,(preamp) for the amplification from instrument level is reasonable to obtain a higher
digital resolution and less quantization noise at the analog-digital conversion.

In addition to that, a suitable voltage bridging is necessary to gain a maximum voltage level on the HAT input. The existing 10\,k$\Omega$ to 25\,k$\Omega$ output impedance from a regular guitar does not ideally match with the 16,753\,k$\Omega$ \cite[p.\,50]{Albers:2017} input impedance provided by the HAT.


% DAS GEHT !!!!
%\begin{wrapfigure}{r}{0.5\textwidth} 
%  \begin{center}
%    \includegraphics[width=0.35\textwidth]{banksy.jpg}
%  \end{center}
%  \caption{A gull}
%\end{wrapfigure}

The MXR-Micro Amp is a simple but efficient classical effect unit designed as a stompbox\,(figure \ref{fig:MicroAmp}).
It is categorized  as a \textit{Boost} effect, which is a clean volume increaser without any modification of the sound. Taking the advantage of available schematics from the internet\,\cite{ElectroSmash:2017} the Micro Amp is used as a base for the preamp development.

\begin{figure}[H]
	\centering \includegraphics[width=0.35\textwidth]{banksy.jpg}
	\caption[MicroAmp]{MXR Micro-Amp stompbox\footnotemark}
	\label{fig:MicroAmp}
\end{figure} 
\footnotetext{URL: https://www.constantinecruz.com/product/mxr-m133-micro-amp-pedal/ [cited 28 August 2018]}

The Micro Amp is designed in a non-inverting topology originally equipped with the operational amplifier TL061 \cite{Tl061:2015} by Texas Instruments. The main advantage of the TL061 is the very low power consumption, necessary for a battery operating stompbox.
A very low power consumption is not needed, so this part is replaced by the low noise TL071ACP, which is often used for high-fidelity  and audio pre-amplifier applications \cite{Tl071:2017}.\\
Table \ref{tab:OPAmpvergleich} shows a comparison in terms of slew rate $SR$, equivalent input 
noise voltage $V_\mathrm{n}$ and THD to justify this choice.


\begin{table}[H]
\begin{center}
\begin{tabular}{|c|c|c|c|}
\hline 
\textbf{operational amplifier} & \textbf{$SR/ \frac{\mathrm{V}}{\mu \mathrm{s}}$} & $V_\mathrm{n} / \frac{\mathrm{nV}}{\sqrt{\mathrm{Hz}}}$ &\textbf{$\mathrm{THD}_{\%}$} \\ 
\hline 
\hline
TL061 & 3,5 & 42 & no information \\ 
\hline 
TL071ACP & 13 & 18 & 0,003 \\ 
\hline 
\end{tabular}
\caption{Comparsion of TL061 and TL071ACP operational amplifier}
\end{center}
\label{tab:OPAmpvergleich}
\end{table}

Besides the pre-amplification to adjust the guitar signal level for the Audio-HAT, the output signal of the HAT needs to be attenuated. The reduction of the voltage level from consumer audio line level back to instrument level is planned with a simple voltage divider.


% Need to match signals... IN 6.3mm Out 3.5mm
% Albers mentioned Line Level -> better SNR
% Impedance matching
% Same loudness


\subsection{User Interface Module}\label{cap:designUI}

On the front panel of the case, the user interaction takes places. 
The Raspberry provides a 40 pin header for the interaction with external components.
23 of the 40 pins are allocated for the communication with the Audio HAT\,\cite[p.\,99]{Albers:2017}. The remaining 17 pins can be used for the interconnecting with the user interface module via general purpose input/output (GPIO).

\subsubsection{Display} \label{DesignDisplay}

In regard to an easy use, a suitable display is required to visualize the current state of the effect unit.
There is a vast amount of liquid crystal displays\,(LCD) available for the Raspberry Pi such as graphical displays, touch displays or dot-matrix text modules.\\
The main selection criterion is the usability during live stage performances. Touch displays are a bad solution due to "sweaty" hands of a guitarist on stage. Graphical displays, with a resolution of 128x64 pixel or more, are not necessary because there are no complex graphics intend to visualize.
For a simple navigation through the effects a text display is sufficient.
Considering the word lengths to be displayed an appropriate resolution is 2x20 characters.
Due to the difficult light conditions on stage, an LED background light is useful.
\\
\\
For an easy communication with the Raspberry Pi the chosen LCD 202A BL\,\cite{LCD:2016} by Display Visions provides an integrated HD44780 controller. The interaction via GPIO is possible and libraries including pre-developed functions are available on the internet.
The 5\,VDC supply voltage for the display is provided via GPIO.

%\begin{figure}[H]
%	\centering \includegraphics[width=0.35\textwidth]{banksy.jpg}
%	\caption[LCDExample]{LCD 202A BL: LCD-Modul, 2x20 \cite{Missing 40}.}
%	\label{fig:LCDExample}
%\end{figure} 


\subsubsection{Buttons and Switch} 

For the Power ON/OFF control of the effect unit a simple 230\,V rocker switch with an
imprinted I for ON and 0 for OFF is chosen. Besides that, two push buttons are required.
One for the effect selection and the other one for shutting down the Pi.
Both are foreseen to be connected via GPIO with the Pi, so the maximum switching voltage is
in the extra-low voltage range (<120\,VDC). For a convenient usage on stage, the chosen buttons provide
a snap-action mechanism for a tactile and audible feedback provoking a hearable "click".

% Buttons "Snap action" good for user caused by feedback
% switched Voltage / "Gewinde"
% / explain this choice

\subsubsection{Rotary Encoders} 

As required, three parameters of each effect should be adjustable by the user.
The modification of a parameter is simply interpreted as an increment or decrement of a value.
The most user-friendly solutions are potentiometers or rotary encoders.
The use of a potentiometer requires an extensive analog-to-digtal conversion. In addition to that, the
potentiometer has an unavoidable memory effect caused by the turning position of the wiper.
This would lead to implementation issues in terms of default values for the parameters.\\

Hence, KY-040 rotary encoders (see figure \ref{fig:RotaryEncoder}) are used to fulfil the requirements.
Already soldered on a small printed circuit board\,(PCB) and equipped with pull-up resistors, the rotary encoders can be directly connected via GPIO.

\begin{figure}[H]
	\centering \includegraphics[width=0.2\textwidth]{banksy.jpg}
	\caption[RotaryEncoder]{KY-040 rotary encoder module\footnotemark}
	\label{fig:RotaryEncoder}
\end{figure} 
\footnotetext{https://alltopnotch.co.uk/wp-content/uploads/imported/6/Rotary-Encoder-Module-KY-040-Brick-Sensor-Clickable-Switch-Arduino-ARM-PIC-UK-231884393106.JPG [cited 31 August 2018]}

\subsection{Power Supply}

According to the requirements stated before, the unit should be connected with the 230\,V electricity grid.
The unit requires a 230\,V connection equipped with a protection-earth contact, caused by metallic electrically conductive parts\,(e.g. screws) touchable from the outside of the case.
\\
For the design of the power switching supply, the significant criteria are the lower supply voltages needed for the built-in components.
The Raspberry Pi has a micro USB connection for the 5.1\,VDC operating voltage, covering the supply for the user interface via GPIO as well.
The MXR Micro-Amp, on which the preamp module development is based, was originally  designed for a 9-volt battery.
The operational amplifier is running with a bias voltage\,(virtual ground).
According to the specification of the TL071ACP \citep{Tl071:2017} the recommended supply voltage for the total $V_{\mathrm{CC}}$ is between 10\,VDC and 30\,VDC.
Therefore the desired power switching supply requires a minimum output voltage of 10\,VDC.
For the further reduction of the voltage level for the Raspberry Pi, a DC-to-DC converter can be used.
\\
\\
The electric power of the Raspberry Pi is about 5.1\,V\,$\cdot$\,2.5\,A = 12.75\,W \cite{Pi:2005}, the low power consumption of the preamp module can be neglected. Therefore the power switching supply requires a minimum output power about 13\,W.

Table \ref{tab:PowerSupplyChoice} shows the chosen power supplies. The power switching supply MW LRS-35-12 by Mean Well with an output power of 35\,W is generously designed. This is reasonable in regard to possible further extensions such as the integration of a power amplifier. The step-down module LM2596 is designed for the direct connection to the power switching supply output, providing 5\,VDC for the Raspberry Pi.
\\


\begin{table}[H]
\begin{center}
\begin{tabular}{|c|c|c|c|}
\hline 
\textbf{Type}  & \textbf{Input Voltage}  & \textbf{Output Voltage}  & \textbf{Output Current}  \\ 
\hline 
\hline
MW LRS-35-12 & 85 - 264\,VAC   & 12\,VDC &  3\,A \\ 
\hline 
LM2596 & 3 - 35\,VDC & Adjustable 1.5 -35\,VDC &  3\,A \\ 
\hline 
\end{tabular} 
\caption{Specification of used power supplies \cite{MWLRS35:2016}, \cite{DCDC:2017}}
\end{center}
\label{tab:PowerSupplyChoice}
\end{table}

\section{Software}

The Raspberry Pi 3\,B (plus Audio Hat) is the core component of the effect unit, responsible for the control of the main
features such as audio processing and user interaction.
It is an advantage that the interacting components are all available during the development phase.
That allows an efficient hardware-near programming. Therefore it is planned that the whole software implementation is realised directly on the Raspberry Pi with mouse, keyboard and monitor attached to it.\\
Hence, cross-compiling from an external computer is not necessary.

\subsubsection{Operating System}

There are many different OS\,(Operating Systems) available for the Raspberry Pi.
The official distribution \textit{Raspbian OS} is in direct competition to other systems like \textit{Windows 10 IoT Core} or \textit{Ubuntu Mate}.
For the choice of the ideal OS for this thesis, the main focus is on user-friendliness and the availability
of relevant information and tutorials. In addition to that, the OS should provide the fundament for a necessary IDE. The Linux Debian based \textit{Raspbian OS} is the most distributed operating system and supported by the Raspberry Pi foundation. Therefore it is chosen for the software developing process.
% Raspbian is easy

\subsubsection{Programming Language}

As a consequence of the required minimum latency of the audio signal, a fast programming language is reasonable. \textit{C} or \textit{C++} are common solutions in terms of time-critical tasks in close-to-hardware programming. The planned programming language for this project is \textit{C}. The decision is made on the basis of significant advantages like available libraries for the display control and GPIO declaration. Moreover, the Audio-HAT demonstration program by Sebastian Albers \cite{Albers:2017}, which provides a good basis for further implementations, is written in \textit{C}.
% C-Language cau
% - Demo from ALbers avialable in C
% - Minimize Latency
% - Libraries of Components are in C
% - Interrupt commom in Mü controler


\subsubsection{Integrated Development Environment}

As a suitable integrated development environment\,(IDE) the freeware \textit{Geany} is chosen.
Running on Raspbian OS, \textit{Geany} is a simple but efficient program supporting a vast amount of programming languages including C. There are several plugins available, like a useful debugger or a syntax highlighting function. In fact, the program operates with very short load times, ideal for a hardware-near programming during test phases.
% Available for Rasp
% IDE Geany
% Freeware
% proggraimn lang include C
% debugger  ans extensions




