%---------------------------------------------------------------------------------------------------
% Einarbeitung
%---------------------------------------------------------------------------------------------------
\newpage
%\part{Anfang}
\chapter{Einarbeitung}
F\"ur den Praktikumsbeginn habe ich mich in Absprache mit meinem Portrix Betreuer Knud M\"uller auf folgende Themen vorbereitet:
\begin{itemize}
\item Linux (Kubuntu)
\item Html5/CSS
\item Json
\item Android Studio
\end{itemize}

W\"ahrend meiner Portrix-Zeit haben sich die Eindr\"ucke vertieft und wurden um folgende Aspekte erweitert:

\begin{itemize}
\item Java
\item Git
\item JavaScript
\end{itemize}

Die generelle Informationsbeschaffung erfolgte in den meisten f\"allen \"uber das Internet. Besonders hilfreich waren dabei:

\begin{itemize}
\item stackoverflow
\item Android Developers
\item Udemy (Online Kursplattform)
\item CodingInFlow (Youtube Channel)
\item CodingWithMitch (Youtube Channel)
\end{itemize}

\section{Linux (Kubuntu)}
Die Firma portrix.net benutzt auf ihren Firmenrechnern Kubuntu als Betriebssystem. Es war also notwendig ein Grundwissen an Terminal-Befehlen (Bash) aufzubauen. Dies wurde als Vorbereitung vor dem Praktikum durch das Youtube-Tutorial "The Complete Linux Course: Beginner to Power User!"\footnote{https://www.youtube.com/watch?v=wBp0Rb-ZJak}  geleistet und war besonders in Verbindung mit der Nutzung von GitLab sehr vorteilhaft. 

\section{Html5/CSS}
"HTML ist eine Abk\"urzung f\"ur Hypertext Markup Language. Man versteht darunter eine Computersprache, mit deren Hilfe man Webseiten im Internet erstellen kann. HTML ist keine Programmiersprache im eigentlich Sinn, sondern vielmehr eine sogenannte Auszeichnungssprache."\footnote{https://www.as-computer.de/wissen/unterschiede-html-und-xml/} Ebenfalls als Vorbereitung auf das Praktikum kennengelernt, kam ich mit HTML5 w\"ahrend der T\"atigkeit bei portrix.net nicht direkt in Kontakt. Durch die \"Ahnlichkeit zu XML war jedoch der Einstieg in das Arbeiten mit den .xml Dateien von Android Studio (AS) sehr viel leichter. Auch das Konzept eines Cascading Style Sheet (CSS) l\"asst sich bei AS wiederfinden. Dadurch sind die Apps besser wartbar und es ist leichter das Layout einheitlich zu gestalten.

\section{Json}
"Die JavaScript Object Notation, kurz JSON, ist ein kompaktes Datenformat in einer einfach lesbaren Textform zum Zweck des Datenaustauschs zwischen Anwendungen."\footnote{https://de.wikipedia.org/wiki/JavaScript\_Object\_Notation/} Die hauseigenen Apps der Firma portrix.net (H2.LIVE und LFI) beziehen aktuelle Daten von einem Server. Diese Daten werden im JSON-Format in den Apps ausgewertet und verarbeitet. Der sichere Umgang mit JSON-Objekten und JSON-Arrays ist daher sehr wichtig. Dies wurde w\"ahrend des Praxissemesters erreicht.

\section{Android Studio}
Als Einstieg in das Arbeiten mit Android Studio (AS) wurde der Udemy Kurs "Der Komplette Android 8 Entwickler Kurs - Erstelle 20+ Apps"\footnote{https://www.udemy.com/der-komplette-android-8-entwickler-kurs/} absolviert.
AS bietet eine Entwicklungsumgebung die sowohl Java als auch Kotlin unterst\"utzt. Da die Android Abteilung bei portrix.net ausschlie\ss{}lich mit Java programmiert, bestand kein Bedarf Kotlin zu lernen.
Ein Android Projekt besteht aus vielen Komponenten, welche als Resultat eine Application (App) generieren. Eine einfache \textit{Hello World}~App besteht zum Beispiel aus der MainActivity.java welche die programmatische Logik enth\"alt und der dazugeh\"origen activity\_main.xml die das graphische Layout definiert.
Ein weitere wichtiger Bestandteil von AS ist der integrierte Emulator der nach den spezifischen Anforderungen des Projekts gestaltet werden kann und zu Testzwecken als ein simuliertes Mobiltelefon auf dem PC fungiert. In der professionellen Software Entwicklung ist es allerdings unerl\"asslich auf einem realen Testdevice (Smartphone oder Tablet) zu testen, da sich Abweichungen zwischen Emulator und Smartphone kaum vermeiden lassen und der haptische Aspekt nur auf dem Testdevice getestet werden kann.
Die M\"oglichkeit AS Projekte mit Versionskontrollsystemen (VCS) zu nutzen spielt ebenfalls eine gro\ss{}e Rolle, da mehrere Entwickler gleichzeitig an den Projekten arbeiten, diese gut aufteilen und dokumentieren k\"onnen. Bei portrix.net wird mit GitLab gearbeitet, worauf im weiteren Verlauf eingegangen wird.

\section{Java}
Java als Objekt orientierte Programmiersprache (OOP) ist das Fundament f\"ur das Arbeiten mit Android Studio bei portrix.net. Die Grundlagen der Objekt orientierten Programmierung die an der HAW vermittelt wurden, konnten dementsprechend erweitert und vertieft werden. Dies geschah zu beginn haupts\"achlich durch Kurse der Online-Plattform Udemy zum Thema OOP, wie zum Beispiel "Java leicht gemacht - Der umfassende Java Einsteigerkurs A-Z"\footnote{https://www.udemy.com/programmieren-lernen-mit-java-ein-kurs-fur-einsteiger/}.  
Mit dem dadurch erlangtem Basiswissen, war es m\"oglich bei der Weiterentwicklung der Firmen Apps beteiligt zu sein.

\section{GitLab}
Wie zuvor erw\"ahnt wird GitLab als VCS-Tool verwendet. "GitLab ist eine Webanwendung zur Versionsverwaltung f\"ur Softwareprojekte auf Basis von Git."\footnote{https://de.wikipedia.org/wiki/GitLab} Um Git zu nutzen speichert man sein Inkrement auf dem lokalen PC mit dem \textit{Commit}-Befehl und der \textit{Commit}-Message welche die \"Anderung beschreibt. Anschlie\ss{}end werden alle gew\"unschten Commits mit dem push-Befehl ins Online-Repository gespeichert. Dadurch k\"onnen mehrere Personen gleichzeitig an einem Projekt arbeiten. Es bietet auch die M\"oglichkeit neue Features in einer parallel laufenden Verzweigung (\textit{Branch}) zu entwickeln und so die Funktionalit\"at der bisherigen Version zu erhalten. 

\section{JavaScript}
"JavaScript (kurz JS) ist eine Skriptsprache, die urspr\"unglich 1995 von Netscape f\"ur dynamisches HTML in Webbrowsern entwickelt wurde, um Benutzerinteraktionen auszuwerten, Inhalte zu ver\"andern, nachzuladen oder zu generieren und so die M\"oglichkeiten von HTML und CSS zu erweitern."\footnote{https://de.wikipedia.org/wiki/JavaScript}
W\"ahrend des Praxissemesters gab es eine Situation in der die H2.LIVE App eine JS Benutzerinteraktion auswerten musste. Hierzu wird im weiteren Verlauf des Berichts eingegangen (Siehe H2.LIVE). Anschlie\ss{}end ergab sich keine M\"oglichkeit die Kenntnisse in JS zu vertiefen.

